\section{Procesando datos en paralelo}

\begin{frame}[t,fragile]{Procesando datos en un vector}
\begin{block}{Elevar al cuadrado}
\begin{lstlisting}
void square_seq(std::vector<int> &v) {
  for (auto &x : v) {
    x *= x;
  }
}
\end{lstlisting}
\end{block}

\mode<presentation>{\vfill\pause}
\begin{block}{Calcular el promedio}
\begin{lstlisting}
double average_seq(const std::vector<int> &v) {
  double s = 0.0;
  for (auto x : v) {
    s += x;
  }
  return s / static_cast<double>(v.size());
}
\end{lstlisting}
\end{block}
\end{frame}

\begin{frame}[t,fragile]{Proyecciones}
\begin{itemize}
  \item Se puede expresar la función cuadrado como una \textmark{proyección}
        con el algoritmo \cppid{std::transform()}
\end{itemize}

\mode<presentation>{\vfill\pause}
\begin{block}{Usando std::transform()}
\begin{lstlisting}
#include <algorithm>

void square_stl(std::vector<int> &v) {
  std::transform(v.begin(), v.end(), v.begin(),
                 [](auto x) { return x * x; });
}
\end{lstlisting}
\end{block}
\end{frame}

\begin{frame}[t,fragile]{Reducciones}
\begin{itemize}
  \item Se puede expresar la suma como una \textmark{reducción}
        con \cppid{std::reduce()}.
\end{itemize}

\mode<presentation>{\vfill\pause}
\begin{block}{Usando std::reduce()}
\begin{lstlisting}
#include <numeric>

double average_stl(const std::vector<int> &v) {
  double s = std::reduce(v.begin(), v.end(), 0.0,
                         [](auto x, auto y) { return x + y; });
  return s / static_cast<double>(v.size());
}
\end{lstlisting}
\end{block}
\end{frame}

\begin{frame}[t,fragile]{Versiones paralelas de la STL}
\begin{itemize}
  \item La mayoría de los algoritmos de la STL admiten un parámetro para indicar
        la política de ejecución:
    \begin{itemize}
      \item \cppid{std::execution:seq}:
            Ejecución secuencial.
      \item \cppid{std::execution:par}:
            Ejecución paralela.
      \item \cppid{std::execution:par\_unseq}:
            Ejecución paralela, vectorizada y/o migración entre hilos.
      \item \cppid{std::execution:unseq}:
            Ejecución vectorizada.
    \end{itemize}

\end{itemize}
\end{frame}

\begin{frame}[t,fragile]{Transformación}
\begin{block}{Transformación}
\begin{lstlisting}
void square_stlpar(std::vector<int> &v) {
  std::transform(std::execution::par,
      v.begin(), v.end(), v.begin(),
                 [](auto x) { return x * x; });
}

void square_stlunseq(std::vector<int> &v) {
  std::transform(std::execution::unseq,
                 v.begin(), v.end(), v.begin(),
                 [](auto x) { return x * x; });
}

void square_stlparunseq(std::vector<int> &v) {
  std::transform(std::execution::par_unseq,
                 v.begin(), v.end(), v.begin(),
                 [](auto x) { return x * x; });
}
\end{lstlisting}
\end{block}
\end{frame}

\begin{frame}[t,fragile]{Reducción}
\begin{block}{Reducción}
\begin{lstlisting}
double average_stlpar(const std::vector<int> &v) {
  double s = std::reduce(std::execution::par, v.begin(), v.end(), 0.0,
                         [](auto x, auto y) { return x + y; });
  return s / static_cast<double>(v.size());
}

double average_stlunseq(const std::vector<int> &v) {
  double s = std::reduce(std::execution::unseq, v.begin(), v.end(), 0.0,
                         [](auto x, auto y) { return x + y; });
  return s / static_cast<double>(v.size());
}

double average_stlparunseq(const std::vector<int> &v) {
  double s = std::reduce(std::execution::par_unseq, v.begin(), v.end(), 0.0,
                         [](auto x, auto y) { return x + y; });
  return s / static_cast<double>(v.size());
}
\end{lstlisting}
\end{block}
\end{frame}

\begin{frame}[t,fragile]{Paralelización con oneTBB}
\begin{itemize}
  \item Espacio de nombres \cppid{oneapi::tbb}.
\begin{lstlisting}
using namespace oneapi::tbb;
\end{lstlisting}

  \item Rango de iteración:
    \begin{itemize}
      \item Varios tipos de rangos en la biblioteca.
      \item Rango dentro de un bloque: \cppid{blocked\_range<T>}.
        \begin{itemize}
          \item \cppid{T} puede ser un tipo númerico o iterador.
        \end{itemize}
      \item Al paralelizar se opera sobre rangos.
    \end{itemize}
\end{itemize}
\end{frame}

\begin{frame}[t,fragile]{Cuadrado paralelo}
\begin{block}{parallel\_for()}
\begin{lstlisting}
void square_tbb(std::vector<int> &v) {
  using namespace oneapi::tbb;
  using range = blocked_range<std::vector<int>::iterator>;
  parallel_for(range{v.begin(), v.end()},
               [](const range &r) {
                 for (auto & x : r) {
                   x *= x;
                 }
               });
}
\end{lstlisting}
\end{block}
\end{frame}

\begin{frame}[t,fragile]{Suma paralela}
\begin{block}{parallel\_reduce()}
\begin{lstlisting}
double average_tbb(const std::vector<int> &v) {
  using namespace oneapi::tbb;
  using range_type = blocked_range<std::vector<int>::const_iterator>;
  double sum = parallel_reduce(
      range_type{v.begin(), v.end()}, 0.0,
      [](const range_type &r, double x) -> double {
        for (double value : r) {
          x += value;
        }
        return x;
      },
      [](auto x, auto y) { return x + y; });
  return sum / static_cast<double>(v.size());
}
\end{lstlisting}
\end{block}
\end{frame}

\begin{frame}[t,fragile]{Cuadrado paralelo + STL}
\begin{block}{parallel\_for()}
\begin{lstlisting}
void square_tbb(std::vector<int> &v) {
  using namespace oneapi::tbb;
  using range = blocked_range<std::vector<int>::iterator>;
  parallel_for(range{v.begin(), v.end()},
               [](const range &r) {
                 std::transform(r.begin(), r.end(), r.begin(),
                                [](auto x) { return x * x; });
               });
}
\end{lstlisting}
\end{block}
\end{frame}

\begin{frame}[t,fragile]{Suma paralela}
\begin{block}{parallel\_reduce()}
\begin{lstlisting}
double average_tbb(const std::vector<int> &v) {
  using namespace oneapi::tbb;
  using range_type = blocked_range<std::vector<int>::const_iterator>;
  double sum = parallel_reduce(
      range_type{v.begin(), v.end()}, 0.0,
      [](const range_type &r, double x) -> double {
        return std::reduce(r.begin(), r.end(), x);
      },
      [](auto x, auto y) { return x + y; });
  return sum / static_cast<double>(v.size());
}
\end{lstlisting}
\end{block}
\end{frame}
