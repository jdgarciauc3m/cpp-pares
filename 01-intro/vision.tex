\section{Visión general}

\begin{frame}[t]{¿Qué es oneTBB?}
\begin{itemize}
  \item \textgood{oneTBB}: One Threading Building Blocks.
    \begin{itemize}
      \item Evolución de Intel Threading Building Blocks.
    \end{itemize}

  \mode<presentation>{\vfill\pause}
  \item Una bibliteca para expresar el paralelismo.

  \mode<presentation>{\vfill\pause}
  \item \textemph{Características}:
    \begin{itemize}
      \item Permite especificar el \textmark{paralelismo lógico} en
            vez de \textbad{trabajar directamente con hilos}.
      \item Se centra el uso de hilos para \textgood{mejorar el rendimiento}.
      \item Pone el foco en la escalabilidad en el \textmark{paralelismo de datos}.
      \item Se apoya fuertemente en la \textmark{programación genérica}.
    \end{itemize}
\end{itemize}
\end{frame}

\begin{frame}[t]{Estructura de oneTBB}
\begin{itemize}

  \item \textmark{Algoritmos y estructuras paralelas}:
    \begin{itemize}
      \item Algoritmos genéricos paralelos.
      \item Grafos de flujo.
      \item Contenedores concurrentes.
    \end{itemize}

  \mode<presentation>{\vfill\pause}
  \item \textmark{Hilos y sincronización}:
    \begin{itemize}
      \item Primitivas de sincronización.
      \item Temporizadores y excepciones.
      \item Hilos.
      \item Almacenamiento local al hilo.
    \end{itemize}

  \mode<presentation>{\vfill\pause}
  \item \textmark{Memoria y planificación}:
    \begin{itemize}
      \item Planifcadores de tareas.
      \item Gestión de memoria.
    \end{itemize}
\end{itemize}
\end{frame}


