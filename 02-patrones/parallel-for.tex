\subsection{parallel\_for()}

\begin{frame}[t,fragile]{Bucles for paralelos}
\begin{itemize}
  \item Útil para transformar bucles con número conocido de iteraciones.
\begin{lstlisting}
for (int i=0; i<max; ++i) {
  f(v[i])
}
\end{lstlisting}

  \mode<presentation>{\vfill\pause}
  \item Ejecuta una función en un espacio de iteraciones.
\begin{lstlisting}
tbb::parallel_for(0, max, [=](int i) {
  f(v[i]);
});
\end{lstlisting}
    \begin{itemize}
      \item El objeto invocable toma un índice como parámetro.
    \end{itemize}
\end{itemize}
\end{frame}

\begin{frame}[t,fragile]{Bucles basados en índice}
\begin{itemize}
  \item Un bucle for basado en índice reparte el espacio de iteraciones
        entre varias tareas.
    \begin{itemize}
\begin{lstlisting}
parallel_for(first, last, func);
\end{lstlisting}
      \item Invoca a \cppid{func(i)} para todo \cppid{i} $\in$ \cppid{[first,last)}.

\mode<presentation>{\pause}
\begin{lstlisting}
parallel_for(first, last, step, func);
\end{lstlisting}
      \item Invoca a \cppid{func(i)} para todo \cppid{i} $\in$ \cppid{[first,last)} 
            con $i=first, first + step, first + 2 \cdot step, \ldots$.
    \end{itemize}

  \mode<presentation>{\vfill\pause}
  \item \textmark{Observaciones}:
    \begin{itemize}
      \item El rango de iteración se divide en bloques que se asignan a tareas.
      \item Cada iteración del bucle puede ejecutar potencialmente en paralelo con otras.
      \item El código de \cppid{func()} debe ser \textemph{thread-safe}.
    \end{itemize}
\end{itemize}
\end{frame}

\begin{frame}[t,fragile]{Dependencias de datos}
\begin{itemize}
  \item Se debe evitar que el cuerpo del bucle presente dependencias de datos.
\begin{lstlisting}
tbb::parallel_for(0, max, [=](int i) {
  v[i] = v[i-1] + 1;
});
\end{lstlisting}
    \begin{itemize}
      \item Posibilidad de \textbad{carreras de datos}.
      \item \textbad{Distinto resultado} dependiendo del 
            \textmark{orden de ejecución} entre iteraciones.
    \end{itemize}
\end{itemize}
\end{frame}

\begin{frame}[t,fragile]{Producto de matrices}
\begin{block}{Implementación secuencial}
\begin{lstlisting}
matrix matrix::operator*(const matrix &m) const {
  CONTRACT_PRE(columns_ == m.rows_);
  matrix result{rows_, m.columns()};
  for (int i = 0; i < rows_; ++i) {
    for (int j = 0; j < m.columns(); ++j) {
      double sum = 0;
      for (int k = 0; k < columns_; ++k) {
        sum += this->operator()(i, j) * m(k, j);
      }
      result(i, j) = sum;
    }
  }
  return result;
}
\end{lstlisting}
\end{block}
\end{frame}

\begin{frame}[t,fragile]{Producto de matrices}
\begin{block}{Implementación paralela}
\begin{lstlisting}[escapechar=@]
matrix matrix::operator*(const matrix &m) const {
  CONTRACT_PRE(columns_ == m.rows_);
  matrix result{rows_, m.columns()};
  @\color{red}{tbb::parallel\_for}@(0, rows_, [*this, m, &result](int i) {
    for (int j = 0; j < m.columns(); ++j) {
      double sum = 0;
      for (int k = 0; k < columns_; ++k) {
        sum += this->operator()(i, j) * m(k, j);
      }
      result(i, j) = sum;
    }
  });
  return result;
}
\end{lstlisting}
\end{block}
\end{frame}


