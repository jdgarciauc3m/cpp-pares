\subsection{parallel\_reduce()}

\begin{frame}[t,fragile]{Reducciones}
\begin{itemize}
  \item Una reducción calcula el resultado de aplicar una
        operación binaria a una secuencia de valores.
\[
red = x_1 \oplus x_2 \oplus \ldots \oplus x_n
\]

    \mode<presentation>{\vfill\pause}
    \begin{itemize}
      \item Se asume que la operación es asociativa:
\[
x_1 \oplus (x_2 \oplus x_3) = (x_1 \oplus x_2) \oplus x_3
\]
      \item Y conmutativa:
\[
x_1 \oplus x_2 = x_2 \oplus x_1
\]

      \mode<presentation>{\vfill\pause}
      \item \textmark{Efecto}:
\[ (x_1 \oplus x_2 \oplus x_3 \oplus x_4) \oplus (x_5 \oplus x_6 \oplus x_7 \oplus x_8) = \]
\mode<presentation>{\vspace{-3em}}
\[ (x_5 \oplus x_6 \oplus x_7 \oplus x_8) \oplus (x_1 \oplus x_2 \oplus x_3 \oplus x_4) \]

    \end{itemize}

\end{itemize}
\end{frame}

\begin{frame}[t,fragile]{Reducción}
\begin{itemize}
  \item Función de reducción paralela:
\begin{lstlisting}
template <typename Range, typename Value, 
          typename Reduction, typename Join>
Value tbb::parallel_reduce(
    const Range & range, const Value & identity, 
    const Reduction & red, const Join & join);
\end{lstlisting}

  \begin{itemize}
    \item \cppid{range}: Rango de iteración.
    \item \cppid{identity}: Valor identidad para la operación de reducción.
    \item \cppid{red}: Operación de reducción entre un rango y un valor inicial.
    \item \cppid{join}: Operación de combinación de resultados parciales.
  \end{itemize}
\end{itemize}
\end{frame}

\begin{frame}[t,fragile]{Rangos}
\begin{itemize}
  \item Un rango representa un \textgood{espacio de iteraciones}, 
        que se puede dividr en \textmark{subrangos}.
    \begin{itemize}
      \item Rango unidimensional: \cppid{blocked\_range}.
    \end{itemize}

  \mode<presentation>{\vfill\pause}
  \item Rangos sobre índices numéricos:
\begin{lstlisting}
blocked_range<int> rng1{-10,10};
blocked_range<std::size_t> rng2{0, vec.size()};
\end{lstlisting}

  \mode<presentation>{\vfill\pause}
  \item Rangos sobre iteradores:
\begin{lstlisting}
blocked_range<std::vector<double>::iterator> rng3{vec.begin(), vec.end()};
\end{lstlisting}

\end{itemize}
\end{frame}

\begin{frame}[t,fragile]{Reducción y combinación}
\begin{itemize}
  \item Reducción sobre un rango:
    \begin{itemize}
      \item Operación sobre un rango y un valor inicial.
    \end{itemize}
\begin{lstlisting}
[=](const Range & rng, Value init) -> Value {
  for (auto i= rng.begin(); i!=rng.end(); ++i) {
    init = mi_operacion(init, vec[i]);
  }
  return init;
}
\end{lstlisting}

  \mode<presentation>{\vfill\pause}
  \item Combinación de resultados parciales:
    \begin{itemize}
      \item Combina dos valores.
    \end{itemize}
\begin{lstlisting}
[](Value x, Value y} -> Value {
  return mi_operacion(x,y);
}
\end{lstlisting}
\end{itemize}
\end{frame} 

\begin{frame}[t,fragile]{Cálculo del máximo con índices}
\begin{block}{Versión secuencial}
\begin{lstlisting}
template <typename T> auto max_value(const std::vector<T> &v) {
  T result = std::numeric_limits<T>::min();
  for (std::size_t i = 0; i < v.size(); ++i) {
    result = std::max(result, v[i]);
  }
  return result;
}
\end{lstlisting}
\end{block}
\end{frame}

\begin{frame}[t,fragile]{Cálculo de máximo con TBB e índices}
\begin{block}{Versión paralela con TBB}
\begin{lstlisting}
template <typename T> auto max_value_reduce_idx(const std::vector<T> &v) {
  return tbb::parallel_reduce(
      tbb::blocked_range<std::size_t>{0, v.size()},
      std::numeric_limits<T>::min(),
      [&](const auto &rng, auto x) {
        for (std::size_t i = rng.begin(); i != rng.end(); ++i) {
          x = std::max(x, v[i]);
        }
        return x;
      },
      [](auto x, auto y) { return std::max(x, y); });
}
\end{lstlisting}
\end{block}
\end{frame}

\begin{frame}[t,fragile]{Cálculo del máximo basado en rango}
\begin{block}{Versión secuencial}
\begin{lstlisting}
template <typename T> auto max_value_rng(const std::vector<T> &v) {
  T result = std::numeric_limits<T>::min();
  for (auto x : v) {
    result = std::max(result, x);
  }
  return result;
}
\end{lstlisting}
\end{block}
\end{frame}

\begin{frame}[t,fragile]{Cálculo del máximo con TBB basado en rango}
\begin{block}{Versión paralela con TBB}
\begin{lstlisting}
template <typename T> auto max_value_reduce_iter(const std::vector<T> &v) {
  return tbb::parallel_reduce(
      tbb::blocked_range{v.begin(), v.end()}, std::numeric_limits<T>::min(),
      [](const auto &rng, auto x) {
        for (auto value : rng) {
          x = std::max(x, value);
        }
        return x;
      },
      [](auto x, auto y) { return std::max(x, y); });
}
\end{lstlisting}
\end{block}
\end{frame}

\begin{frame}[t,fragile]{Cálculo del máximo con STL paralela}
\begin{block}{Versión paralela con TBB}
\begin{lstlisting}
template <typename T> auto max_value_stl_par(const std::vector<T> &v) {
  return std::reduce(std::execution::par, v.begin(), v.end(),
                     std::numeric_limits<T>::min(),
                     [](auto x, auto y) { return std::max(x, y); });
}
\end{lstlisting}
\end{block}
\begin{itemize}
  \item Otras alternativas:
    \begin{itemize}
      \item \cppid{std::execution::unseq}
      \item \cppid{std::execution::par\_unseq}
    \end{itemize}
\end{itemize}
\end{frame}

\begin{frame}[t]{Cálculo de $\pi$}
\begin{itemize}
  \item Cálculo de $\pi$:
\[
\int_0^1 \frac{1}{1+x^2} = arctg(1) - arctg(0) = \frac{\pi}{4}
\]

  \item Aproximación:
\[
\pi \approx 4 \cdot \sum_{i=1}^N f(\frac{i}{N} + \frac{\Delta x}{2} ) \Delta x
\qquad;\qquad 
\Delta x = \frac{1}{N}
\]
\[
\pi \approx 4 \cdot \sum_{i=1}^N f(\frac{i}{N} + \frac{1}{2N}) \frac{1}{N} = 
4 \cdot \sum_{i=1}^N f(\frac{i+0.5}{N}) \frac{1}{N}
\]
\[
\pi \approx \frac{4}{N} \cdot \sum_{i=1}^N \frac{1}{1 + (\frac{i+0.5}{N})^2}
\]

\end{itemize}
\end{frame}

\begin{frame}[t,fragile]{Versión secuencial}
\begin{block}{Cálculo de pi}
\begin{lstlisting}
double compute_pi_seq(long n) {
  const double deltax = 1.0 / static_cast<double>(n);
  double sum = 0;
  for (long i=0; i<n; ++i) {
    const double x = (static_cast<double>(i) + 0.5) * deltax;
    sum += 1.0 / (1.0 + square(x));
  }
  return 4 * sum * deltax;
}
\end{lstlisting}
\end{block}
\end{frame}

\begin{frame}[t,fragile]{Versión paralela con TBB}
\begin{block}{Cálculo de pi con parallel\_reduce}
\begin{lstlisting}
double compute_pi_tbb(long n) {
  const double deltax = 1.0 / static_cast<double>(n);
  double sum = tbb::parallel_reduce(
      tbb::blocked_range<long>{0,n},
      0.0,
      [=](const tbb::blocked_range<long> & range, double init) {
        for (long i=range.begin(); i!=range.end(); ++i) {
          const double x = (static_cast<double>(i) + 0.5) * deltax;
          init += 1.0 / (1.0 + square(x));
        }
        return init;
      },
      [](auto x, auto y) { return x+y; }
  );
  return 4 * sum * deltax;
}
\end{lstlisting}
\end{block}
\end{frame}
