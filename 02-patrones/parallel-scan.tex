\subsection{parallel\_scan()}

\begin{frame}[t]{Suma prefija}
\begin{itemize}
  \item Una \textemph{suma prefija} calcula un conjunto de 
        \textmark{sumas parciales} a partir de una secuencia de datos.
\[y_1 = x_1\]
\[y_2 = x_1 \oplus x_2\]
\[y3 = x_1 \oplus x_2 \oplus x_3\]
\[\dots\]
\[y_n = x_1 \oplus x_2 \oplus \ldots \oplus x_n\]
\end{itemize}
\end{frame}

\begin{frame}[t,fragile]{Scan paralelo}
\begin{itemize}
  \item Función paralela de \emph{scan}:
\begin{lstlisting}
template <typename Range, typename Value,
          typename Scan, typename ReverseJoin>
Value tbb::parallel_scan(
    const Range & range, const Value & identity,
    const Scan & scan, const ReverseJoin & join);                   
\end{lstlisting}
  \begin{itemize}
    \item \cppid{range}: Rango de iteración.
    \item \cppid{identity}: Valor de identidad para la operación de scan.
    \item \cppid{scan}: Operación de scan.
      \begin{itemize}
        \item Puede ejecutarse más de una vez sobre el mismo subrango: 
              \textmark{pre-scan} y \textmark{final-scan}.
      \end{itemize}
    \item \cppid{join}: Operación de combinación de resultados parciales.
  \end{itemize}
\end{itemize}
\end{frame}

\begin{frame}[t,fragile]{Operación de scan}
\begin{itemize}
  \item Scan de un subrango:
    \begin{itemize}
      \item Un tercer parámetro para indicar \textmark{pre-scan} versus \textmark{final-scan}.
    \end{itemize}
\end{itemize}
\begin{lstlisting}
[=,&result](const Range & rng, Value init, bool is_final) {
  for (auto i=rng.begin(); i!=rng.end(); ++i) {
    init = mi_operacion(init,vec[i]);
    if (is_final) { result[i] = init; }
  }
  return init;
}
\end{lstlisting}
\end{frame}

\begin{frame}[t,fragile]{Cálculo de una suma prefija}
\begin{block}{Versión secuencial}
\begin{lstlisting}
auto prefix_sum(const std::vector<int> &v) {
  const std::size_t max = v.size();
  std::vector<int> result;
  result.reserve(max);
  result.push_back(v[0]);
  for (std::size_t i = 1; i < max; ++i) {
    result.push_back(result[i - 1] + v[i]);
  }
  return result;
}
\end{lstlisting}
\end{block}
\end{frame}

\begin{frame}[t,fragile]{Cálculo de una suma prefija}
\begin{block}{Versión paralela}
\begin{lstlisting}
auto prefix_sum_tbb(const std::vector<int> &v) {
  const std::size_t max = v.size();
  std::vector<int> result(max);
  tbb::parallel_scan(
      tbb::blocked_range<std::size_t>{1, max}, int{0},
      [v, &result](const tbb::blocked_range<std::size_t> &rng, int sum,
                   bool is_final) {
        for (auto i = rng.begin(); i != rng.end(); ++i) {
          sum += v[i];
          if (is_final) {
            result[i] = sum;
          }
        }
        return sum;
      },
      [](auto x, auto y) { return x + y; });
  return result;
}
\end{lstlisting}
\end{block}
\end{frame}
